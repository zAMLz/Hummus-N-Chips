 
%%  _   _ _   _ __  __ __  __ _   _ ____    ____  _    _   _ ____  
%% | | | | | | |  \/  |  \/  | | | / ___|  |  _ \| |  | | | / ___| 
%% | |_| | | | | |\/| | |\/| | | | \___ \  | |_) | |  | | | \___ \ 
%% |  _  | |_| | |  | | |  | | |_| |___) | |  __/| |__| |_| |___) |
%% |_| |_|\___/|_|  |_|_|  |_|\___/|____/  |_|   |_____\___/|____/ 
%%

%% Set the document class, we will using the IEEEtran class
\documentclass[12pt,journal,compsoc]{article}

%% Import certain useful packages
\usepackage{graphicx}
\usepackage{minted}
\usepackage{amsmath}
\usepackage{amsfonts}
\usepackage{hyperref}
\usepackage{array}


%% Time for the main attraction
\begin{document}

\title{HummusPlus~Language~Specifications}
\author{Amlesh~Sivanantham}
\date{}

% What is the point of this part exactly?
\markboth{HummusPlus~Language~Specifications}
{Amlesh Sivanantham: Hummus-N-Chips}

%Start the abstract
% \IEEEcompsoctitleabstractindextext{%
% \begin{abstract}

% \end{abstract}}

% Make the title

\begin{titlepage}
  \begin{center}
    \vfill
    %\includegraphics[width=0.15\textwidth]{example-image-1x1}\par\vspace{1cm}
    {\scshape\LARGE Hummus-N-Chips \par}
    \vspace{1cm}
    {\scshape\Large The Hummus and Chips collection of Compiler, Assembler, 
                    and Simulator\par}
    \vspace{1.5cm}
    {\huge\bfseries HummusPlus~Language Specifications\par}
    \vspace{2cm}
    {\Large\itshape Amlesh Sivanantham\par}
    \vspace{1cm}
    {\large \today\par}
  \end{center}
  % Bottom of the page
\end{titlepage}

\clearpage

\maketitle

\clearpage
\begin{center}
\vspace{1.5cm}
\scalebox{0.60}{
\begin{tabular}{ | p{3.0cm} | m{2.05cm} | p{5.0cm} | p{12.0cm} | }
  \hline
  \textbf{INSTRUCTION} & \textbf{OPCODE} & \textbf{PSUEDO-CODE} & \textbf{DESCRIPTION} \\ 
  \hline \hline
  %--------------------------------------------------
  HALT  & 0~~0~~0~~0 & \mintinline{python}|exit()| & Ends the program. \\
  \hline 
  %--------------------------------------------------
  SHFF  & 0~~0~~0~~1 & \mintinline{python}|PC += unsigned(ARG)| & Moves the Program Counter Forward \\
  \hline 
  %--------------------------------------------------
  SHFB  & 0~~0~~1~~0 & \mintinline{python}|PC -= unsigned(ARG)| & Moves the Program Counter Backward \\
  \hline 
  %--------------------------------------------------
  BNR   & 0~~0~~1~~1 & \inputminted{python}{.docfiles/bnr.txt} & If the result is not zero, add the signed number to the Program Counter, else do nothing. \\
  \hline 
  %--------------------------------------------------
  INP   & 0~~1~~0~~0 & \inputminted{python}{.docfiles/inp.txt} & If the value of ARG is not zero, store the user input into B2, else store it into B1. \\
  \hline 
  %--------------------------------------------------
  STR   & 0~~1~~0~~1 & \inputminted{python}{.docfiles/str.txt} & If the value of ARG is not zero, store the value of RESULT into B2, else store it into B1. \\
  \hline 
  %--------------------------------------------------
  LDB1  & 0~~1~~1~~0 & \mintinline{python}|B1 = Mem[ARG]| & Take the byte from location ARG in Data Memory and store it into B1. \\
  \hline 
  %--------------------------------------------------
  LDB2  & 0~~1~~1~~1 & \mintinline{python}|B2 = Mem[ARG]| & Take the byte from location ARG in Data memory and store it into B2. \\
  \hline 
  %--------------------------------------------------
  ADDB1 & 1~~0~~0~~0 & \mintinline{python}|RS=B1+unsigned(ARG)| & Add the unsigned value of ARG to register B1 and store it into RESULT. \\
  \hline 
  %--------------------------------------------------
  ADDB2 & 1~~0~~0~~1 & \mintinline{python}|RS=B2+unsigned(ARG)| & Add the unsigned value of ARG to register B2 and store it into RESULT. \\
  \hline 
  %--------------------------------------------------
  BOOL  & 1~~0~~1~~0 & . . . & Perform a Boolean operation on B1 and B2 based on ARG and store it into RESULT. More details below. \\
  \hline 
  %--------------------------------------------------
  ADD   & 1~~0~~1~~1 & . . . & Perform a Addition operation on B1 and B2 based on ARG and store it into RESULT. More details below. \\
  \hline 
  %--------------------------------------------------
  SUBB1 & 1~~1~~0~~0 & \mintinline{python}|RS=B1-unsigned(ARG)| & Subtract the unsigned value of ARG from register B1 and store it into RESULT. \\
  \hline 
  %--------------------------------------------------
  SUBB2 & 1~~1~~0~~1 & \mintinline{python}|RS=B2-unsigned(ARG)| & Subtract the unsigned value of ARG from register B2 and store it into RESULT. \\
  \hline 
  %--------------------------------------------------
  STM   & 1~~1~~1~~0 & \mintinline{python}|Mem[ARG] = RS| & Store the value of B1 or B2 in location RESULT in Data Memory. \\
  \hline 
  %--------------------------------------------------
  MEM   & 1~~1~~1~~1 & . . . & Handle clearing of Memory or Dynamic storing or reading of Data Memory based on value of ARG. \\
  \hline
\end{tabular}}

~~~~~~~~~~~~

\scalebox{0.60}{
\begin{tabular}{ | p{5.5cm} | p{5.0cm} | m{12.0cm} | }
  \hline
  \hline
  \textbf{ADDITION ARGUMENT} & \textbf{PSUEDO-CODE} & \textbf{DESCRIPTION} \\
  \hline
  \hline
  %--------------------------------------------------
  ~~~~~~0~~0~~0~~0 & \mintinline{python}|RS = B1 + B2| & Add B1 and B2. \\
  \hline
  %--------------------------------------------------
  ~~~~~~0~~1~~0~~0 & \mintinline{python}|RS = B1 - B2| & Subtract B2 from B1. \\
  \hline
  %--------------------------------------------------
  ~~~~~~1~~0~~0~~0 & \mintinline{python}|RS = -B1 + B2| & Subtract B1 from B2. \\
  \hline
  %--------------------------------------------------
  ~~~~~~1~~1~~0~~0 & \mintinline{python}|RS = -B1 - B2| & Subtract B2 from Negative B1. \\
  \hline
\end{tabular} 
}

~~~~~~~~~~~~~

\scalebox{0.60}{
\begin{tabular}{ | p{5.5cm} | p{5.0cm} | m{12.0cm} | }
  \hline
  \hline
  \textbf{BOOLEAN ARGUMENT} & \textbf{PSUEDO-CODE} & \textbf{DESCRIPTION} \\
  \hline
  \hline
  %--------------------------------------------------
  ~~~~~~0~~0~~0~~0 & \mintinline{python}|RS = B1 AND B2| & Perform a bitwise AND on B1 and B2 and store it into RESULT. \\
  \hline
  %--------------------------------------------------
  ~~~~~~0~~0~~0~~1 & \mintinline{python}|RS = B1 L-AND B2| & Perform a logical AND on B1 and B2 and store it into RESULT. \\
  \hline
  %--------------------------------------------------
  ~~~~~~0~~0~~1~~0 & \mintinline{python}|RS = B1 OR B2| & Perform a bitwise OR on B1 and B2 and store it into RESULT. \\
  \hline
  %--------------------------------------------------
  ~~~~~~0~~0~~1~~1 & \mintinline{python}|RS = B1 L-OR B2| & Perform a logical OR on B1 and B2 and store it into RESULT. \\
  \hline
  %--------------------------------------------------
  ~~~~~~0~~1~~0~~0 & \mintinline{python}|RS = B1 ^ B2| & Perform a bitwise XOR on B1 and B2 and store it into RESULT. \\
  \hline
  %--------------------------------------------------
  ~~~~~~0~~1~~0~~1 & \mintinline{python}|RS = B1 ~^ B2| & Perform a bitwise XNOR on B1 and B2 and store it into RESULT. \\
  \hline
  %--------------------------------------------------
  ~~~~~~0~~1~~1~~0 & \mintinline{python}|RS = B1 << 1| & Bitshift to the left by 1 on B1 and store it into RESULT. \\
  \hline
  %--------------------------------------------------
  ~~~~~~0~~1~~1~~1 & \mintinline{python}|RS = B2 << 1| & Bitshift to the left by 1 on B2 and store it into RESULT. \\
  \hline
  %--------------------------------------------------
  ~~~~~~1~~0~~0~~0 & \mintinline{python}|RS = ~(B1 AND B2)| & Perform a bitwise NAND on B1 and B2 and store it into RESULT. \\
  \hline
  %--------------------------------------------------
  ~~~~~~1~~0~~0~~1 & \mintinline{python}|RS = ~(B1 L-AND B2)| & Perform a logical NAND on B1 and B2 and store it into RESULT.\\
  \hline
  %--------------------------------------------------
  ~~~~~~1~~0~~1~~0 & \mintinline{python}|RS = ~(B1 OR B2)| & Perform a bitwise NOR on B1 and B2 and store it into RESULT. \\
  \hline
  %--------------------------------------------------
  ~~~~~~1~~0~~1~~1 & \mintinline{python}|RS = ~(B1 L-OR B2)| & Perform a logical NOR on B1 and B2 and store it into RESULT. \\
  \hline
  %--------------------------------------------------
  ~~~~~~1~~1~~0~~0 & \mintinline{python}|RS = ~B1| & Perform a bitwise NOT on B1 and store it into RESULT. \\
  \hline
  %--------------------------------------------------
  ~~~~~~1~~1~~0~~1 & \mintinline{python}|RS = ~B2| & Perform a bitwise NOT on B2 and store it into RESULT. \\
  \hline
  %--------------------------------------------------
  ~~~~~~1~~1~~1~~0 & \mintinline{python}|RS = B1 >> 1| & Bitshift to the right by 1 on B1 and store it into RESULT. \\
  \hline
  %--------------------------------------------------
  ~~~~~~1~~1~~1~~1 & \mintinline{python}|RS = B2 >> 1| & Bitshift to the right by 1 on B2 and store it into RESULT. \\
  \hline
\end{tabular} 
}

~~~~~~~~~~~~~

\scalebox{0.60}{
\begin{tabular}{ | p{5.5cm} | p{5.0cm} | m{12.0cm} | }
  \hline
  \hline
  \textbf{MEMORY ARGUMENT} & \textbf{PSUEDO-CODE} & \textbf{DESCRIPTION} \\
  \hline
  \hline
  %--------------------------------------------------
  ~~~~~~0~~0~~0~~0 & \mintinline{python}|For all Mem, Mem = 0| & Clear the data memory. \\
  \hline
  %--------------------------------------------------
  ~~~~~~0~~0~~0~~1 & \mintinline{python}|B1 = Mem[B1]| & Take byte from location B1 and store it into B1.\\
  \hline
  %--------------------------------------------------
  ~~~~~~0~~0~~1~~0 & \mintinline{python}|B1 = Mem[B2]| & Take byte from location B2 and store it into B1. \\
  \hline
  %--------------------------------------------------
  ~~~~~~0~~0~~1~~1 & \mintinline{python}|B1 = Mem[RS]| & Take byte from location RESULT and store it into B1. \\
  \hline
  %--------------------------------------------------
  ~~~~~~0~~1~~0~~0 & \mintinline{python}|Mem[B1] = B1| & Store value of B1 in address B1 in Data Memory. \\
  \hline
  %--------------------------------------------------
  ~~~~~~0~~1~~0~~1 & \mintinline{python}|B2 = Mem[B1]| & Take byte from location B1 and store it into B2. \\
  \hline
  %--------------------------------------------------
  ~~~~~~0~~1~~1~~0 & \mintinline{python}|B2 = Mem[B2]| & Take byte from location B2 an store it into B2. \\
  \hline
  %--------------------------------------------------
  ~~~~~~0~~1~~1~~1 & \mintinline{python}|B2 = Mem[RS]| & Take byte from location RESULT and store it into B2. \\
  \hline
  %--------------------------------------------------
  ~~~~~~1~~0~~0~~0 & \mintinline{python}|Mem[B1] = B2| & Store value of B2 in address B1 in Data Memory. \\
  \hline
  %--------------------------------------------------
  ~~~~~~1~~0~~0~~1 & \mintinline{python}|Mem[B2] = B1| & Store value of B1 in address B2 in Data Memory. \\
  \hline
  %--------------------------------------------------
  ~~~~~~1~~0~~1~~0 & \mintinline{python}|Mem[B2] = B2| & Store value of B2 in address B2 in Data Memory. \\
  \hline
  %--------------------------------------------------
  ~~~~~~1~~0~~1~~1 & \mintinline{python}|Mem[B2] = RS| & Store value of REULT in address B2 in Data Memory. \\
  \hline
  %--------------------------------------------------
  ~~~~~~1~~1~~0~~0 & \mintinline{python}|Mem[B1] = RS| & Store value of RESULT in address B1 in Data Memory. \\
  \hline
  %--------------------------------------------------
  ~~~~~~1~~1~~0~~1 & \mintinline{python}|Mem[RS] = B1| & Store value of B1 in address RESULT in Data Memory. \\
  \hline
  %--------------------------------------------------
  ~~~~~~1~~1~~1~~0 & \mintinline{python}|Mem[RS] = B2| & Store value of B2 in address RESULT in Data Memory. \\
  \hline
  %--------------------------------------------------
  ~~~~~~1~~1~~1~~1 & \mintinline{python}|Mem[RS] = RS| & Store value of RESULT in address RESULT in Data Memory. \\
  \hline
\end{tabular} 
}

\end{center}

\clearpage
Example of the language can be seen below. This is the implementation of a Turing Machine.

\inputminted{python}{.docfiles/turingMachine.hal}

\end{document}